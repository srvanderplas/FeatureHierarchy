\documentclass[12pt]{article}
% \usepackage{geometry}                % See geometry.pdf to learn the layout options. There are lots.
% \geometry{letterpaper}                   % ... or a4paper or a5paper or ... 
%\usepackage{graphicx}
\usepackage{subcaption}
\usepackage{afterpage}
\usepackage{amssymb}
\usepackage{natbib}
\usepackage{amsmath}
\usepackage{amsfonts}
% \usepackage{color}
\usepackage{multirow}
\usepackage{rotating}
\usepackage[dvipsnames,svgnames,table]{xcolor}
\usepackage{hyperref}
\graphicspath{{figure/}}

\DeclareGraphicsRule{.tif}{png}{.png}{`convert #1 `dirname #1`/`basename #1 .tif`.png}
\usepackage[colorinlistoftodos]{todonotes}
%---------------------------------------------------
%                 Editing Commands
\newcommand{\done}[2][inline]{\todo[color=SpringGreen, #1]{#2}}  % for todos that have been seen and dealt with
\newcommand{\meh}[2][inline]{\todo[color=White, #1]{#2}}   % for todos that may no longer be relevant 
\newcommand{\comment}[2][inline]{\todo[color=SkyBlue, #1]{#2}} % for comments that may not be "to-do"s
%\newcommand{\mcomment}[1]{\todo[color=SkyBlue]{#1}} % for margin comments
\newcommand{\newtext}[1]{\todo[inline, color=White]{ \color{OliveGreen}{#1}}} % new text - not necessarily something to be done
\newcommand{\newdo}[1]{\todo[inline, color=Lime]{#1}} % new to do item
%
%---------------------------------------------------
%                 Placing Figures


%---------------------------------------------------
% Define new environment
\newtheorem{theorem}{Theorem}[section]
\newtheorem{algorithm}[theorem]{Algorithm}
%---------------------------------------------------

%\pdfminorversion=4
% NOTE: To produce blinded version, replace "0" with "1" below.
\newcommand{\blind}{0}

% DON'T change margins - should be 1 inch all around.
\addtolength{\oddsidemargin}{-.5in}%
\addtolength{\evensidemargin}{-.5in}%
\addtolength{\textwidth}{1in}%
\addtolength{\textheight}{1.3in}%
\addtolength{\topmargin}{-.8in}%


\begin{document}

%\bibliographystyle{natbib}

\def\spacingset#1{\renewcommand{\baselinestretch}%
{#1}\small\normalsize} \spacingset{1}


%%%%%%%%%%%%%%%%%%%%%%%%%%%%%%%%%%%%%%%%%%%%%%%%%%%%%%%%%%%%%%%%%%%%%%%%%%%%%%

\if0\blind
{
  \title{\bf Clusters beat Trend!? \\Testing feature hierarchy in statistical graphics}
  \author{Susan VanderPlas\thanks{
    The authors gratefully acknowledge \textit{please remember to list all relevant funding sources in the unblinded version}}\hspace{.2cm}\\
    Department of Statistics and Statistical Laboratory, Iowa State University\\
    and \\
    , Heike Hofmann\\
    Department of Statistics and Statistical Laboratory, Iowa State University}
  \maketitle
} \fi

\if1\blind
{
  \bigskip
  \bigskip
  \bigskip
  \begin{center}
    {\LARGE\bf Clusters beat Trend!? \\Testing feature hierarchy in statistical graphics}
\end{center}
  \medskip
} \fi

\bigskip
\begin{abstract}
Graphics are very effective for communicating numerical information quickly and efficiently, but many of the design choices we make are based on subjective measures, such as personal taste or conventions of the discipline rather than objective criteria. We briefly introduce perceptual principles such as preattentive features and gestalt heuristics, and then discuss the design and results of a factorial experiment designed to examine the effect of plot aesthetics such as color and trend lines on participants' assessment of ambiguous data displays. The quantitative and qualitative experimental results strongly suggest that plot aesthetics have a significant impact on the perception of important features in data displays. 
\end{abstract}

\noindent%
{\it Keywords:}  3 to 6 keywords, that do not appear in the title
\vfill

\newpage
\spacingset{1.45} % DON'T change the spacing!

\tableofcontents
\newpage
\section{Introduction and Background}
Numerical information can be difficult to communicate effectively in raw form, due to limits on attention span, short term memory, and information storage mechanisms within the human brain. 
Graphics are much more effective for communicating numerical information, as (well-designed) graphics order the numerical information spatially and utilize the higher-bandwidth visual system. 
Visual data displays serve as a form of external cognition \citep{zhang1997nature,scaife1996external}, ordering and visually summarizing data which would be hopelessly confusing in tabular format. 
One fantastic example of this phenomenon is the Hertzsprung-Russell (HR) diagram, which was described as ``one of the greatest observational syntheses in astronomy and astrophysics" because it allowed astronomers to clearly relate the absolute magnitude of a star to its' spectral classification; facilitating greater understanding of stellar evolution \citep{spence1993remarkable}. 
The data it displayed was previously available in several different tables; when plotted on the same chart, information that was invisible in a tabular representation became immediately clear \citep{lewandowsky1989perception}. 
Graphical displays more efficiently utilize cognitive resources by reducing the burden of storing, ordering, and summarizing raw data; this frees bandwidth for higher levels of information synthesis, allowing observers to note outliers, understand relationships between variables, and form new hypotheses.

Graphical displays are powerful because they efficiently and effectively convey numerical information, but there exists  relatively sparse empirical information about how the human perceptual system processes these displays. Our understanding of the perception of statistical graphics is informed by general psychological and psychophysics research as well as more specific research into the perception of data displays \citep{cleveland:1984}. 

One relevant focus of psychological research is pre-attentive perception, that is, perception which occurs automatically in the first 200 ms of exposure to a visual stimulus \citep{treisman1985preattentive}. 

Research into {\bf preattentive perception} provides us with some information about the temporal hierarchy of graphical feature processing. Color, line orientation, and shape are processed preattentively; that is, within 200 ms, it is possible to identify a single target in a field of distractors, if the target differs with respect to color or shape \citep{goldstein2009encyclopedia}. 
Research by \citet{healey1999large} extends this work, demonstrating that certain features of three-dimensional data displays are also processed preattentively. However, neither target identification nor three-dimensional data processing always translate into faster or more accurate inference about the data displayed, particularly when participants have to integrate several preattentive features to understand the data. 

{\bf Feature detection} at the attentive stage of perception has also been examined in the context of statistical graphics; researchers have evaluated the perceptual implications of utilizing color, fill, shapes, and letters to denote categorical or stratified data in scatterplots. \citet{cleveland:1984} ranked the optimality of these plot aesthetics based on response accuracy, preferring colors, amount of fill, shapes, and finally letters to indicate category membership. \citet{lewandowsky1989discriminating} examined both accuracy and response time, finding that color is faster and more accurately perceived (except by individuals with color deficiency). Shape, fill, and discriminable letters (letters which do not share visual features, such as HQX) were identified as less accurate than color, while confusable letters (such as HEF) result in significantly decreased accuracy. 

{\bf Gestalt psychology} is another area of psychological research, that examines perception as a holistic experience, establishing and evaluating mental heuristics used to transform visual stimuli into useful, coherent information. 
Gestalt rules of perception can be easily applied to statistical graphics, as they describe the way we organize visual input, focusing on the holistic experience rather than the individual perceptual features. 

For example, rather than perceiving four legs, a tail, two eyes, two ears, and a nose, we perceive a dog. This is due to certain perceptual heuristics, which provide a ``top-down" method of understanding visual stimuli by taking into account past experience. 

The rules of perceptual organization relevant to graphical perception in this experiment are:
\begin{itemize}
\item \textbf{Proximity}: two elements which are close together are more likely to belong to a single unit.
\item \textbf{Similarity}: the more similar two elements are, the more likely they belong to a single unit.
\item \textbf{Good continuation}: two elements which blend together smoothly likely belong to one unit.
\item \textbf{Common region}: elements contained within a common region likely belong together. 
\end{itemize}
A complete list of the rules of perceptual grouping can be found in \citet{goldstein2009encyclopedia}.


\begin{figure}\centering
\begin{knitrout}
\definecolor{shadecolor}{rgb}{0.969, 0.969, 0.969}\color{fgcolor}

{\centering \includegraphics[width=0.32\linewidth]{figure/gestalt1-1} 
\includegraphics[width=0.32\linewidth]{figure/gestalt1-2} 
\includegraphics[width=0.32\linewidth]{figure/gestalt1-3} 

}



\end{knitrout}
\caption[Gestalt principles applied to statistical plots]{\label{fig:gestalt} \emph{Proximity} renders the fifty points of the first scatterplot as two distinct (and equal-sized) groups. Shapes and colors create different groups of points in the middle scatterplot, invoking the Gestalt principle of \emph{Similarity}. \emph{Good Continuation} renders the points in the scatterplot on the right hand side into two groups of points on curves: one a straight line with an upward slope, the other a curve that initially decreases and at the end of the range shows an uptick.} 
\end{figure}
% \afterpage{\clearpage}

The plots in Figure~\ref{fig:gestalt} demonstrate several of the gestalt principles which combine to order our perceptual experience from the top down. These laws help to order our perception of charts as well: points which are colored or shaped the same are perceived as belonging to a group (similarity), points within a bounding interval or ellipse are perceived as belonging to the same group (common region), and regression lines with confidence intervals are perceived as single units (continuity and common region). 

The processing of visual stimuli utilizes low-level feature detection, which occurs automatically in the preattentive perceptual phase, and higher-level mental heuristics which are informed by experience. Both types of mental processes utilize physical location, color, and shape  to organize perceptual stimuli and direct attention to graphical features which stand out. 

Research on preattentive perception is important because features that are perceived preattentively do not require as much mental effort to process from raw visual stimuli; subsequent top-down gestalt heuristics can be applied to the categorized features in order to make sense of the visual scene once the attentive stage of perception is reached. 


This paper describes the results of a user study designed to explore the hierarchy of gestalt principles in perception of statistical graphics. We utilize information from previous studies \citep{heer:2014, robinson:03, healey1996high} concerning the hierarchy of preattentive feature perception in order to maximize the effect of preattentive feature differences. 


Statistical graphics can be difficult to examine experimentally; qualitative studies rely on descriptions of the plot by participants who may not be able to articulate their observations precisely, while quantitative studies may only be able to examine whether the viewer can accurately read numerical information from the chart, instead of exploring the overall utility of the data display holistically. Here, we are describing the setup and results of a study using statistical lineup methodology to provide quantitative and qualitative information.

\paragraph{Statistical lineups} 
are an important experimental tool for evaluating the perceptual utility of graphical displays. Lineups fuse commonly used psychological tests (target identification, visual search) \citep{visualreasoning} with statistical hypothesis tests to facilitate formal experimental evaluation of statistical graphics. 

Lineups are an experimental tool designed to serve as a visual hypothesis test, separating ``significant" visual effects from those that would be expected under a null hypothesis \citep{buja2009statistical, majumder2013validation,hofmann2012graphical, wickham2010graphical}. 
A statistical lineup consists of (usually) 20 sub-plots, arranged in a grid (examples are shown in Figure~\ref{fig:plotExamples}). 
Of these plots, one plot is the ``target plot'', generated from either real data or an alternate model (equivalent to $H_A$ in hypothesis testing); the other 19 plots are generated either using bootstrap samples of the real data or by generating ``true null" plots from the null distribution $H_0$. 
If participants can identify the target plot from the field of distractors, then the visual display is deemed significant in the same sense that a numerical test with $p<0.05$ is significant. 

Apart from the hypothesis testing construct, the use of statistical lineups to test statistical graphics conforms nicely to psychological testing constructs such as visual search \citep{demita1981validity,treisman1980feature}, where a single target is embedded in a field of distractors and response time, accuracy, or both are used to measure the complexity of the underlying psychological processes leading to identification. 

% \comment{Should the next 3 paragraphs go into a different section? Otherwise, the "Intro" part in the last paragraph gets hidden under the "Statistical Lineups" heading...}

In this paper we {\bf modify the lineup protocol} by introducing a second target to each lineup. The two targets represent two different, competing signals; an observer's choice then demonstrates empirically which signal is more salient. 
If both targets exhibit similar signal, observers may identify both targets, removing any forced-choice scenario which might skew results in a study. % (few participants exercised this option). 

By tracking the proportion of observers choosing either target plot (a measure of overall lineup difficulty) as well as which proportion of observers choose one target over the other target, we can determine the relative strength of the two competing signals amid a field of distractors. At this level, signal strength is determined by the experimental data and the generating model; we are measuring the ``power" (in a statistical sense) of the human perceptual system, rather than raw numerical signal. 

Using this testing framework, we  apply different aesthetics, such as color and shape, as well as plot objects which display statistical calculations, such as trend lines and bounding ellipses. These additional plot layers, discussed in more detail in the next section, are designed to emphasize one of the two competing targets and affect the overall visual signal of the target plot relative to the null plots. We expect that in a situation similar to the third plot of Figure~\ref{fig:gestalt}, the addition of two trend lines would emphasize the ``good continuation" of points in the plot, producing a stronger visual signal, even though the underlying data has not changed. Similarly, the grouping effect in the first plot in the Figure~should be enhanced if the points in each group were colored differently, as the proximity heuristic is supplemented by similarity. In plots that are ambiguous, containing some clustering of points as well as a linear relationship between $x$ and $y$, additional aesthetic cues may ``tip the balance" in favor of recognizing one type of signal.

The study in this paper is designed to inform our understanding of the perceptual implications of these additional aesthetics, in order to provide guidelines for the creation of data displays which provide visual cues consistent with gestalt heuristics and preattentive perceptual preferences. 

The next section discusses the particulars of the experimental design, including the data generation model, plot aesthetics, selection of color and shape palettes, and other important considerations. Experimental results are presented in section \ref{sec:Results}, and implications and conclusions are discussed in section \ref{sec:Conclusion}. 

\section{Experimental Setup and Design} \label{sec:ExperimentalDesign}
In this section, we discuss the generating data models for the two types of signal plots and the null plots, the selection of plot aesthetic combinations and aesthetic values, and the design and execution of the experiment.

\subsection{Data Generation}
Lineups require a single ``target" data set (which we are expanding to two competing ``target" data sets), and a method for generating null plots. When utilizing real data for target plots, null plots are often generated through permutations.

Here, it is possible to generate true null plots, which are generated from the null model and do not depend on the data used in the target plot. 
This experiment will measure two competing gestalt heuristics, proximity and good continuation, using two data-generating models: $M_C$, which generates data with $K$ clusters, and $M_T$, which generates data with a positive correlation between $x$ and $y$. 
True null datasets are created using a mixture model $M_0$ which combines $M_C$ and $M_T$. Both $M_C$ and $M_T$ generate data in the same range of values. 
Additionally, $M_C$ generates clustered data with linear correlations that are within $\rho = (0.25, 0.75)$, similar to the linear relationship between datasets generated by $M_0$, and $M_T$ generates data with clustering similar to $M_0$. These constraints provide some assurance that participants who select a plot with data generated from $M_T$ are doing so because of visual cues indicating a linear trend (rather than a lack of clustering compared to plots with data generated from $M_0$), and participants who select a plot with data generated from $M_C$ are doing so because of visual cues indicating clustering, rather than a lack of a linear relationship relative to plots with data generated from $M_0$. 


\subsubsection{Regression Model \texorpdfstring{$M_T$}{Mt}}
This model has the parameter $\sigma_T$ to reflect the amount of scatter around the trend line. It generates $N$ points $(x_i, y_i), i=1, ..., N$ where $x$ and $y$ have a positive linear relationship. The data generation mechanism is as follows: 

\begin{algorithm}\hfill\newline
  Input Parameters: sample size $N$, $\sigma_T$ standard deviation around the line \\
  Output: $N$ points, in form of vectors $x$ and $y$.
  \begin{enumerate}
    \item Generate $\tilde{x}_i$, $i=1, ..., N$, as a sequence of evenly spaced points from $[-1, 1]$. 
    \item Jitter $\tilde{x}_i$ by adding small uniformly distributed perturbations to each of the values: $x_i = \tilde{x}_i + \eta_i$, where $\eta_i \sim \text{Unif}(-z, z)$, $z = \frac{2}{5(N-1)}$.
    \item Generate $y_i$ as a linear regressand of $x_i$: $y_i = x_i + e_i$, $e_i \sim N(0, \sigma^2_T)$.
    \item Center and scale $x_i$, $y_i$.
  \end{enumerate}
\end{algorithm}

We compute the coefficient of determination for all of the plots to assess the amount of linearity in each panel, computed as 
\begin{equation}\label{eq:linearMeasure}
R^2 = 1 - \frac{RSS}{TSS},
\end{equation}
where TSS is the total sum of squares, $TSS = \sum_{i=1}^N \left(y_i - \bar{y}\right)^2$ and $RSS = \sum_{i=1}^N e_i^2$, the residual sum of squares.
The expected value of the coefficient of determination $E\left[R^2\right]$ in this scenario is 
\[
E\left[R^2\right] =  \frac{1}{1 + 3\sigma^2_T},
\]
because
$E[RSS] = N\sigma^2_T$ and $E[TSS] = \sum_{i=1}^N E\left[y_i^2\right]$  (as $E[Y] = 0$), where 
$$
E\left[y_i^2\right] = E\left[x_i^2 + e_i^2 + 2 x_ie_i\right] = \frac{1}{3} + \sigma^2_T. 
$$
The use of $R^2$ to assess the strength of the linear relationship (rather than the correlation) is indicated because human perception of correlation strength more closely aligns with $R^2$ \citep{bobko1979perception,lewandowsky1989perception}. 

\begin{figure}[ht]
\begin{knitrout}
\definecolor{shadecolor}{rgb}{0.969, 0.969, 0.969}\color{fgcolor}\begin{kframe}


{\ttfamily\noindent\bfseries\color{errorcolor}{\#\# Error in eval(expr, envir, enclos): could not find function "{}ldply"{}}}

{\ttfamily\noindent\bfseries\color{errorcolor}{\#\# Error in paste("{}sigma[T] :"{}, res\$sd.trend): object 'res' not found}}

{\ttfamily\noindent\bfseries\color{errorcolor}{\#\# Error in ggplot(data, aesthetics, environment = env): object 'res' not found}}\end{kframe}
\end{knitrout}
\caption[Parameters affecting $M_T$]{\label{fig:trends} Set of scatterplots showing one draw each from the trend model $M_T$ for parameter values of  $\sigma_T \in \{0.1, 0.2, 0.3, 0.4\}$.}
\end{figure}

\subsubsection{Cluster Model \texorpdfstring{$M_C$}{Mc}} 
We begin by generating $K$ cluster centers on a $K \times K$ grid, then we generate points around selected cluster centers. 
\begin{algorithm}\hfill\newline
  Input Parameters:  $N$ points, $K$ clusters, $\sigma_C$ cluster standard deviation \\
  Output: $N$ points, in form of vectors $x$ and $y$. 
  \begin{enumerate}
    \item Generate cluster centers $(c^x_{i}, c^y_{i})$ for each of the $K$ clusters, $i=1, ..., K$:
      \begin{enumerate}
        \item in form of two vectors $c^{x}$ and $c^y$ of permutations of $\{1, ..., K\}$, such that
        \item the correlation between cluster centers \text{cor}$(c^{x}, c^{y})$ falls into a range of $[.25, .75]$.
      \end{enumerate}
      \item Center and standardize cluster centers $(c^x, c^y)$:  
      \[
        \tilde{c}^x_{i} = \frac{c^x_{i} - \bar{c}}{s_c} \ \ \text{ and } \ \ \tilde{c}^y_{i} = \frac{c^y_{i} - \bar{c}}{s_c},
      \]
      where $\overline{c} = (K+1)/2$ and $s_c^2 = \frac{K(K+1)}{12}$ for all $i = 1, ..., K$.
    \item For the $K$ clusters, we want to have nearly equal sized groups, but allow some variability. Cluster sizes  $g = (g_1, ..., g_K)$ with $N = \sum_{i=1}^K g_i$, for clusters $1, ..., K$ are therefore determined as a draw from a multinomial distribution: 
    \[
    g \sim \text{Multinomial }(K, p) \text{ where } p = \tilde{p}/\sum_{i=1}^K \tilde{p}_i, \text{ for } \tilde{p} \sim N \left(\frac{1}{K}, \frac{1}{2 K^2} \right).
    \]
     
    \item Generate points around cluster centers by adding small normal perturbations: 
      \begin{eqnarray*}
        x_i &=& \tilde{c}^x_{g_i} + e^x_i, \text{ where } e^x_i \sim N(0, \sigma^2_C),\\
        y_i &=& \tilde{c}^y_{g_i} + e^y_i, \text{ where } e^y_i \sim N(0, \sigma^2_C).
      \end{eqnarray*}
    \item Center and scale $x_i$, $y_i$.
  \end{enumerate}
\end{algorithm} 

\begin{figure}[bht]
\begin{knitrout}
\definecolor{shadecolor}{rgb}{0.969, 0.969, 0.969}\color{fgcolor}\begin{kframe}


{\ttfamily\noindent\bfseries\color{errorcolor}{\#\# Error in eval(expr, envir, enclos): could not find function "{}ldply"{}}}

{\ttfamily\noindent\bfseries\color{errorcolor}{\#\# Error in res\$K <- 3: object 'res' not found}}

{\ttfamily\noindent\bfseries\color{errorcolor}{\#\# Error in eval(expr, envir, enclos): object 'res' not found}}

{\ttfamily\noindent\bfseries\color{errorcolor}{\#\# Error in eval(expr, envir, enclos): object 'res' not found}}

{\ttfamily\noindent\bfseries\color{errorcolor}{\#\# Error in eval(expr, envir, enclos): could not find function "{}ldply"{}}}

{\ttfamily\noindent\bfseries\color{errorcolor}{\#\# Error in res2\$K <- 5: object 'res2' not found}}

{\ttfamily\noindent\bfseries\color{errorcolor}{\#\# Error in eval(expr, envir, enclos): object 'res2' not found}}

{\ttfamily\noindent\bfseries\color{errorcolor}{\#\# Error in eval(expr, envir, enclos): object 'res2' not found}}

{\ttfamily\noindent\bfseries\color{errorcolor}{\#\# Error in rbind(res, res2): object 'res' not found}}

{\ttfamily\noindent\bfseries\color{errorcolor}{\#\# Error in paste("{}sigma[C] :"{}, res\$sd.cluster): object 'res' not found}}

{\ttfamily\noindent\bfseries\color{errorcolor}{\#\# Error in paste("{}K :"{}, res\$K): object 'res' not found}}

{\ttfamily\noindent\bfseries\color{errorcolor}{\#\# Error in ggplot(aes(x = x, y = y, color = color, shape = shape), data = res): object 'res' not found}}\end{kframe}
\end{knitrout}
\caption[Parameters affecting $M_C$]{\label{fig:clusters} Scatterplots of clustering output for different inner cluster spread $\sigma_C$  (left to right) and different number of clusters $K$ (top and bottom), generated using the same random seed at each parameter setting. The colors and shapes shown are those used in the lineups for $K=3$ and $K=5$.}
\end{figure}
As a measure of cluster cohesion we use a coefficient to assess the amount of variability within each cluster, compared to total variability. Note that for the purpose of clustering, variability is measured as the variability in both $x$ and $y$ from a common mean, i.e.\ we implicitly assume that the values in $x$ and $y$ are on the same scale. This ensures that $\sigma_C$is a scaling parameter that regulates the amount of cluster cohesion (see Figure~\ref{fig:clusters}).  % (which we achieve by scaling in the final step of the generation algorithm).


%\afterpage{\clearpage}

\comment{$\sigma_C$ is the theoretical regulator of variability, while $C^2$ is the after-the-fact measure of the amount of spread. XXX Do we know the relationship between those two measurements? -- it should be similar to $R^2$ versus $\sigma$.}
For two numeric variables $x$ and $y$ and grouping variable $g$ with $g_i \in \{1, ..., K\}, i = 1, ..., n$, we compute the  {\it cluster index} $C^2$ as follows: let $j(i)$ be the function that maps index $i = 1, ..., n$ to one of the clusters $1, ..., K$ given by the grouping variable $g$. Then for each  level of $g$, we find  a cluster center as $\bar{x}_{j(i)}$ and  $\bar{y}_{j(i)}$, and we determine the strength of the clustering by comparing the within cluster variability with the overall variability: 

\begin{eqnarray}\label{eq:clusterMeasure}
C^2 &=& \frac{CSS}{TSS},\\
\nonumber CSS &=& \sum_{i=1}^n \left(x_{j(i)} - \overline{x}_{j(i)}\right)^2 + \left(y_{j(i)} - \overline{y}_{j(i)} \right)^2, \\
\nonumber TSS &=& \sum_{i=1}^n \left(x_i - \bar{x}\right)^2 + \left(y_i - \bar{y}\right)^2.
\end{eqnarray}


\subsubsection{Null Model \texorpdfstring{$M_0$}{M0}}
The generative model for null data is a mixture model $M_0$ that draws $n_c \sim \text{Binomial}(N, \lambda)$ observations from the cluster model, and $n_T = N - n_c$ from the regression model $M_T$. Observations are assigned to specific clusters using hierarchical clustering, which creates groups consistent with any structure present in the generated data. This provides a plausible grouping for use in aesthetic and statistics requiring categorical data (color, shape, bounding ellipses). 

\begin{figure}[hbt]
\begin{knitrout}
\definecolor{shadecolor}{rgb}{0.969, 0.969, 0.969}\color{fgcolor}\begin{kframe}


{\ttfamily\noindent\bfseries\color{errorcolor}{\#\# Error in eval(expr, envir, enclos): could not find function "{}ldply"{}}}

{\ttfamily\noindent\bfseries\color{errorcolor}{\#\# Error in res\$K <- 3: object 'res' not found}}

{\ttfamily\noindent\bfseries\color{errorcolor}{\#\# Error in eval(expr, envir, enclos): object 'res' not found}}

{\ttfamily\noindent\bfseries\color{errorcolor}{\#\# Error in eval(expr, envir, enclos): object 'res' not found}}

{\ttfamily\noindent\bfseries\color{errorcolor}{\#\# Error in eval(expr, envir, enclos): could not find function "{}ldply"{}}}

{\ttfamily\noindent\bfseries\color{errorcolor}{\#\# Error in res2\$K <- 5: object 'res2' not found}}

{\ttfamily\noindent\bfseries\color{errorcolor}{\#\# Error in eval(expr, envir, enclos): object 'res2' not found}}

{\ttfamily\noindent\bfseries\color{errorcolor}{\#\# Error in eval(expr, envir, enclos): object 'res2' not found}}

{\ttfamily\noindent\bfseries\color{errorcolor}{\#\# Error in rbind(res, res2): object 'res' not found}}

{\ttfamily\noindent\bfseries\color{errorcolor}{\#\# Error in paste("{}lambda :"{}, res\$lambda): object 'res' not found}}

{\ttfamily\noindent\bfseries\color{errorcolor}{\#\# Error in paste("{}K :"{}, res\$K): object 'res' not found}}

{\ttfamily\noindent\bfseries\color{errorcolor}{\#\# Error in ggplot(aes(x = x, y = y, color = color, shape = shape), data = res): object 'res' not found}}\end{kframe}
\end{knitrout}
\caption[Mixing parameter for null model $M_0$]{\label{fig:lambda} Scatterplots of data generated from $M_0$ using different values of $\lambda$, generated using the same random seed at each $\lambda$ value.}
\end{figure}
%\afterpage{\clearpage}

Null data in this experiment is generated using $\lambda = 0.5$, that is, each point in a null data set is equally likely to have been generated from $M_C$ and $M_T$. 

\subsubsection{Parameters used in Data Generation}
Models $M_C$, $M_T$, and $M_0$ provide the foundation for this experiment; by manipulating cluster standard deviation $\sigma_C$ and regression standard deviation $\sigma_T$ (directly related to correlation strength) for varying numbers of clusters $K=3, 5$, we can systematically control the statistical signal present in the target plots and generate corresponding null plots that are mixtures of the two distributions. For each parameter set $\{K, N, \sigma_C, \sigma_T\}$, as described in table \ref{tab:parameters}, we  generate a lineup dataset consisting of one set drawn from $M_C$, one set drawn from $M_T$, and 18 sets drawn from $M_0$. 

\begin{table}[hbtp]
  \rowcolors{2}{gray!25}{white}
\begin{center}
\begin{tabular}{lll}
\bf Parameter & \bf Description & \bf Choices\\\hline
$K$ & \# Clusters &  \begin{tabular}{l}3, 5 \end{tabular} \\
$N$ & \# Points &  \begin{tabular}{l}$15\cdot K$\end{tabular} \\
$\sigma_T$ & Scatter around trend line &   \begin{tabular}{l}.15, .25, .35  \end{tabular}\\
$\sigma_C$ & Scatter around cluster centers & \begin{tabular}{ll} .25, .30, .35 ($K=3$)\\ .20, .25, .30 ($K=5$) \end{tabular}
\\\hline
\end{tabular}
\end{center}
\caption{Parameter settings for generation of lineup datasets. \label{tab:parameters}}
\end{table}

The parameter values were chosen after examining the full parameter space through simulation of 1000 lineup datasets for each combination of $\sigma_T\in\{0.2, 0.25, ..., 0.5\}$, $\sigma_C\in\{0.1, 0.15, ..., 0.4\}$, and $K\in\{3,5\}$; 
for each data set generated, the previously described statistics for trend and cluster strength were computed. We compared the statistics for the relevant target plot to the most extreme value for the 18 null plots. 

These distributions allow us to objectively assess the difficulty of detecting the target datasets computationally (without relying on human perception). A target plot with $R^2=0.95$ is very easy to identify when surrounded by null plots with $R^2=0.5$, while null plots with $R^2=0.9$ make the target plot more difficult to identify. This approach is similar to that taken in \citet{niladri:2014}. 

Figure~\ref{fig:targetsignal-0} shows  densities of each measure computed from the  maximum of 18 null plots compared to the measure in the signal plot for one combination of parameters.
There is some overlap in the distribution of $R^2$ for the null plots compared to the target plot displaying data drawn from $M_T$. As a result, the distribution of the cluster statistic values are more easily separated from the null data sets than the distribution of the line statistic, that is, $\sigma_C = 0.20$ is producing cluster target data sets that are a bit easier to identify numerically than trend targets with a parameter value of $\sigma_T = 0.25$.

\begin{figure}[ht]
\centering
\begin{knitrout}
\definecolor{shadecolor}{rgb}{0.969, 0.969, 0.969}\color{fgcolor}

{\centering \includegraphics[width=\maxwidth]{figure/null-distribution-1-1} 

}



\end{knitrout}
\caption[Simulation-based test statistic density for null and target plots]{\label{fig:targetsignal-0}Density of test statistics measuring trend strength and cluster strength for target distributions and null plots based on 1,000 draws of lineup data with $\sigma_T= 0.25, \sigma_C=0.20$ and $K=3$. }
\end{figure}
%\afterpage{\clearpage}

%\newdo{move simulation results back in here.}
Graphical summaries of simulation results for a whole range of values for $\sigma_C$ and $\sigma_T$ are provided in appendix \ref{app:parametersimulation}. Using information from the simulation, we identified values of $\sigma_T$ and $\sigma_C$ corresponding to ``easy", ``medium" and ``hard" numerical comparisons between corresponding target data sets and null data sets. It is important to note that the numerical measures we have described in equations \eqref{eq:linearMeasure} and \eqref{eq:clusterMeasure} only provide information on the numerical discriminability of the target datasets from the null datasets; the simulation cannot provide us with information on the perceptual discriminability, and it has been established that human perception of scatterplots does not replicate statistical measures exactly \citep{bobko1979perception, mosteller1981eye, lewandowsky1989perception}.

Each of the generated datasets is then plotted as a lineup, where we apply aesthetics which emphasize clusters and/or linear relationships, to experimentally determine how these aesthetics change participants' ability to identify each target plot. The next section describes the aesthetic combinations and their anticipated effect on participant responses. 

\subsection{Lineup Rendering}
\subsubsection{Plot Aesthetics}
Gestalt perceptual theory suggests that perceptual features such as shape, color, trend lines, and boundary regions modify the perception of ambiguous graphs, emphasizing clustering in the data (in the case of shape, color, and bounding ellipses) or linear relationships (in the case of trend lines and prediction intervals), as demonstrated in Figure~\ref{fig:gestalt}. For each dataset we examine the effect of plot aesthetics (color, shape) and statistical layers (trend line, boundary ellipses, prediction intervals) shown in table \ref{tab:plotaesthetics}  on target identification. Examples of these plot aesthetics are shown in Figure~\ref{fig:plotExamples}.

\begin{figure}[ht]
\centering
\begin{subfigure}[t]{0.25\linewidth}
  \caption{Plain}\vspace{-0.15in}
  \includegraphics[width=\linewidth]{figure/fig-samplepics-1}
\end{subfigure}
\begin{subfigure}[t]{0.25\linewidth}
  \caption{Color}\vspace{-0.15in}
  \includegraphics[width=\linewidth]{figure/fig-samplepics-2}
\end{subfigure}
\begin{subfigure}[t]{0.25\linewidth}
  \caption{Shape}\vspace{-0.15in}
  \includegraphics[width=\linewidth]{figure/fig-samplepics-3}
\end{subfigure}
\begin{subfigure}[t]{0.25\linewidth}
  \caption{Shape + Color}\vspace{-0.15in}
  \includegraphics[width=\linewidth]{figure/fig-samplepics-4}
\end{subfigure}
\begin{subfigure}[t]{0.25\linewidth}
  \caption{Color + Ellipse}\vspace{-0.15in}
  \includegraphics[width=\linewidth]{figure/fig-samplepics-5}
\end{subfigure}
\begin{subfigure}[t]{0.25\linewidth}
  \caption{Shape + Ellipse}\vspace{-0.15in}
  \includegraphics[width=\linewidth]{figure/fig-samplepics-6}
\end{subfigure}
\begin{subfigure}[t]{0.25\linewidth}
  \caption{Trend}\vspace{-0.15in}
  \includegraphics[width=\linewidth]{figure/fig-samplepics-7}
\end{subfigure}
\begin{subfigure}[t]{0.25\linewidth}
  \caption{Trend + Error }\vspace{-0.15in}
  \includegraphics[width=\linewidth]{figure/fig-samplepics-8}
\end{subfigure}
\begin{subfigure}[t]{0.25\linewidth}
  \caption{Trend + Color}\vspace{-0.15in}
  \includegraphics[width=\linewidth]{figure/fig-samplepics-9}
\end{subfigure}
\begin{subfigure}[t]{0.25\linewidth}
  \caption{Trend + Color + Ellipse}\vspace{-0.15in}
  \includegraphics[width=\linewidth]{figure/fig-samplepics-10}
\end{subfigure}
\caption[Sample lineup stimuli for each of the 10 aesthetic combinations]{Each of the 10 plot feature combinations tested in this study, with $K=3$, $\sigma_T=0.25$ and $\sigma_C=0.20$. \label{fig:plotExamples}}
\end{figure}
%\afterpage{\clearpage}


\begin{table}[ht]
\centering
\scalebox{0.8}{
\begin{tabular}{ccccc}
\begin{tabular}{c} \phantom{.}\\ \phantom{.} \end{tabular} && \multicolumn{3}{c}{\cellcolor{gray!25} Line Emphasis} \\
& Strength & 0 & 1 & 2 \\
\cellcolor{gray!25}\begin{tabular}{c} \phantom{.}\\ \phantom{.} \end{tabular} & 0 &  \cellcolor{gray!5} None &  \cellcolor{gray!15} Line &  \cellcolor{gray!25} Line + Prediction \\
\cellcolor{gray!25}\begin{tabular}{c} \\ Cluster \end{tabular} & 1 &  \cellcolor{gray!15}\begin{tabular}{c}Color\\ Shape\end{tabular} & \cellcolor{gray!5} Color + Line \\
\cellcolor{gray!25}\begin{tabular}{c}  Emphasis\\ \phantom{.} \end{tabular} & 2 & \cellcolor{gray!25}\begin{tabular}{c} Color + Shape\\ Color + Ellipse \end{tabular} && \cellcolor{gray!5}\begin{tabular}{c} Color + Ellipse +\\
Line + Prediction \end{tabular}\\
\cellcolor{gray!25}\begin{tabular}{c} \phantom{.}\\ \phantom{.} \end{tabular} & 3 & \cellcolor{gray!35} Color + Shape + Ellipse 
\end{tabular}}
\caption[Aesthetics affecting perception of statistical plots]{Plot aesthetics and statistical layers which impact perception of statistical plots, according to gestalt theory. \label{tab:plotaesthetics}}
\end{table}

\afterpage{\clearpage}

We expect that relative to a plot with no extra aesthetics or statistical layers, the addition of color, shape, and 95\% boundary ellipses  increases the probability of a participant selecting the target plot with data generated from $M_C$, the cluster model, and that the addition of these aesthetics  decreases the probability of a participant selecting the target plot with data generated from $M_T$, the trend model. 

Similarly, we expect that relative to a plot with no extra aesthetics or statistical layers, the addition of a trend line and prediction interval  increases the probability of a participant selecting the target plot with data generated from $M_T$, the trend model, and decreases the probability of a participant selecting the target plot with data generated from $M_C$, the cluster model.

\subsubsection{Color and Shape Palettes}
Colors and shapes used in this study were selected in order to maximize preattentive feature differentiation. \citet{heer:2014} provide sets of 10 colors and 10 shapes, with corresponding distance matrices, determined by user studies. Using these perceptual kernels for shape and color, we identified sets of 3 and 5 colors and shapes which maximize the sum of pairwise differences, subject to certain constraints imposed by software and accessibility concerns. 

\begin{figure}[bhtp]\centering
\begin{subfigure}[t]{0.475\textwidth}
\begin{knitrout}
\definecolor{shadecolor}{rgb}{0.969, 0.969, 0.969}\color{fgcolor}

{\centering \includegraphics[width=\linewidth]{figure/color-palette-1} 

}



\end{knitrout}
\caption[Color palette used to maximize preattentive perception]{Color Palette. For the present study  gray was removed from the palette to make the experiment more inclusive of participants with colorblindness.\label{fig:colors}}
\end{subfigure}
\hfill
\begin{subfigure}[t]{0.475\textwidth}
\begin{knitrout}
\definecolor{shadecolor}{rgb}{0.969, 0.969, 0.969}\color{fgcolor}

{\centering \includegraphics[width=\linewidth]{figure/shape-palette-1} 

}



\end{knitrout}
\caption[Shape palette used to maximize preattentive perception]{Shape palette. Due to varying point size between Unicode vs. non-Unicode characters, the last two shapes were removed for our investigation.\label{fig:shapes}}
\end{subfigure}
\caption{Color and shape palettes investigated for differentiability in \protect\citet{heer:2014}. }
\end{figure}
%\afterpage{clearpage}

The color palette used in \citet{heer:2014} and shown in Figure~\ref{fig:colors} is derived from colors available in Tableau visualization software~\citep{tableau}. 
In order to produce experimental stimuli accessible to the approximately 4\% of the population with red-green color deficiency \citep{colorvision}, we removed the gray hue from the palette. This modification produced maximally different color combinations which did not include red-green combinations, while also removing a color (gray) which is difficult to distinguish for those with color deficiency.  

Software compatibility issues led us to exclude two shapes used in \citet{heer:2014} and shown in Figure~\ref{fig:shapes}. The left and right triangle shapes (available only in unicode within R) were excluded from our investigation due to size differences between unicode and non-unicode shapes. After optimization over the sum of all pairwise distances, the maximally different shape sequences for the 3 and 5 cluster datasets also conform to the guidelines in \citet{robinson:03}: for $K=3$ the shapes are from Robinson's group 1, 2, and 9, for $K=5$ the shapes are from groups 1, 2, 3, 9, and 10. Robinson's groups are designed so that shapes in different groups show differences in preattentive properties; that is, they are easily distinguishable. In addition, all shapes are non-filled shapes, which means that they are consistent with one of the simplest solutions to overplotting of points in the tradition of \citet{tukey, cleveland:85} and \citet{few}. For this reason we abstained from the additional use of alpha-blending of points to diminish the effect of overplotting in the plots.




\subsection{Experimental Design}
The study is designed hierarchically, as a factorial experiment for combinations of $\sigma_C$, $\sigma_T$, and $K$, with three replicates at each parameter combination. These parameters are used to generate lineup datasets which serve as blocks for the plot aesthetic level of the experiment; each dataset is rendered with every combination of aesthetics described in table \ref{tab:plotaesthetics}. Participants are assigned to generated plots according to an augmented balanced incomplete block scheme: each participant is asked to evaluate 10 plots, which consist of one plot at each combination of $\sigma_C$ and $\sigma_T$, randomized across levels of $K$, with one additional plot providing replication of one level of $\sigma_C\times\sigma_T$. Each of a participant's 10 plots will present a different aesthetic combination.

\subsection{Hypotheses}
The primary purpose of this study is to understand how visual aesthetics affect signal detection in the presence of competing signals. We expect that plot modifications which emphasize similarity and proximity, such as color, shape, and 95\% bounding ellipses, will increase the probability of detecting the clustering relationship, while plot modifications which emphasize good continuation, such as trend lines and prediction intervals, will increase the probability of detecting the linear relationship. 

A secondary purpose of the study is to relate signal strength (as determined by dataset parameters $\sigma_C$, $\sigma_T$, and $K$) to signal detection in a visualization by a human observer.

\subsection{Participant Recruitment}
Participants were recruited using Amazon's Mechanical Turk service\citep{amazon}, which connects interested workers with ``Human Intelligence Tasks" (HITs), which are (typically) short tasks which cannot be easily automated. Only workers with at least 100 previous HITs at a 95\% successful completion rate were allowed to sign up for completing the task. These restrictions reduce the amount of data cleaning required by ensuring that participants have experience with the Mechanical Turk system. 

Participants were asked to complete an example task similar to the task in the experiment before deciding whether or not to complete the HIT. The lineups used as examples contained only one target (5 trend and 5 cluster trials were provided), and participants had to correctly identify target plots in at least two lineups before being allowed into  the HIT and proceeding to the experimental phase. The webpage used to collect data from Amazon Turk participants is available at \url{http://www.mlcape.com:8080/mahbub/turk16/index.html}. No data was recorded from the example task because participants had not yet provided informed consent. 

Once participants completed the example task and provided informed consent, they could accept the HIT through Amazon and were directed to the main experimental task. 
Participants were required to complete 10 lineups, answering ``Which plot is the most different from the others?". Participants were asked to provide a short reason for their choice, such as ``Strong linear trend" or ``Groups of points", and to rate their confidence in their selection from 1 (least confident) to 5 (most confident). 
After the first question, basic demographic information was collected: age range, gender, and highest level of education. 

\section{Results}\label{sec:Results}
\newdo{I think we need to straighten out the analysis: after the overview of the demographics, 
the first part should be an assessment of the difficulty of the lineup (using visual p-values, I'll get that started ***update*** visual p-values are all essentially zero, so no need to report anything - but we can include a discussion on how to get visual p-values (simulation based, in the appendix?)), 
so let's go straight into the face-off. 
But in the faceoff we need to add a discussion on how many values we have for evaluation (the CI intervals keep getting bigger as we have more clustering) - for this, we can include the discussion that is in the single target clustering models right now and go into the wordles afterwards (participant reasoning).
}






















